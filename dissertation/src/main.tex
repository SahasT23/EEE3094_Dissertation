\documentclass[a4paper,12pt]{article}

\usepackage{url}
\usepackage{parskip} 	
\usepackage{tabularx}

%other packages for formatting
\RequirePackage{color}
\RequirePackage{graphicx}
\usepackage[usenames,dvipsnames]{xcolor}
\usepackage[scale=0.9]{geometry}  % Removed to avoid duplicate geometry settings.
\usepackage{amsmath}
\usepackage{tikz}
\usetikzlibrary{shapes,arrows}
\usepackage{rotating}
\usetikzlibrary{shapes.geometric, arrows}
\usetikzlibrary{positioning, shapes}
\usepackage{float}
\usepackage{algorithm}
\usepackage{algpseudocode}
\usepackage{tikz}
\usetikzlibrary{arrows.meta, positioning}
\usepackage{setspace}
\usepackage{titlesec}
\usepackage{tikz}
\usepackage{tabularx}
\usepackage{circuitikz}
\usepackage{pgfplots}
\usepackage{amssymb}
\usepackage{amsthm}
\usepackage{float}  
\usepackage{booktabs}
\usepackage{xcolor}
\pgfplotsset{compat=1.18}
\usepackage{setspace}
\usepackage{colortbl}
\usepackage{tcolorbox}
\usepackage{wrapfig}
\usepackage{media9}
\usepackage{siunitx}

% Define colors
\definecolor{blueblock}{RGB}{173, 216, 230}  % Light blue
\definecolor{greenblock}{RGB}{144, 238, 144} % Light green
\definecolor{redblock}{RGB}{255, 182, 193}   % Light red
\definecolor{lightblue}{RGB}{173,216,230}
\definecolor{lightgreen}{RGB}{144,238,144}
\definecolor{lightorange}{RGB}{255,200,150}
\definecolor{lightpurple}{RGB}{220,180,240}

\tikzstyle{startstop} = [rectangle, rounded corners, minimum width=3cm, minimum height=1cm,text centered, draw=black, fill=red!30]
\tikzstyle{process} = [rectangle, minimum width=3cm, minimum height=1cm, text centered, draw=black, fill=blue!30]
\tikzstyle{decision} = [diamond, minimum width=3cm, minimum height=1cm, text centered, draw=black, fill=green!30]
\tikzstyle{arrow} = [thick,->,>=stealth]
\tikzstyle{io} = [trapezium, trapezium left angle=70, trapezium right angle=110, minimum width=3cm, minimum height=1cm, text centered, draw=black, fill=blue!30]

% Define styles with smaller dimensions and font size
\tikzset{
    font=\small,
    startstop/.style={
        rectangle,
        rounded corners,
        minimum width=2.5cm,
        minimum height=0.8cm,
        draw=black,
        fill=red!30,
        text centered,
        % Set text width to wrap text automatically
        text width=3.0cm,
        align=center
    },
    process/.style={
        rectangle,
        minimum width=2.5cm,
        minimum height=0.8cm,
        draw=black,
        fill=orange!30,
        text centered,
        % Set text width to wrap text automatically
        text width=3.0cm,
        align=center
    },
    decision/.style={
        diamond,
        aspect=2,
        draw=black,
        fill=green!30,
        inner sep=0pt,
        minimum width=2.0cm,
        minimum height=1.0cm,
        % Allow text wrapping in decision blocks
        text width=2.2cm,
        align=center
    },
    io/.style={
        trapezium,
        trapezium left angle=70,
        trapezium right angle=110,
        minimum width=2.5cm,
        minimum height=0.8cm,
        draw=black,
        fill=blue!30,
        text centered,
        % Set text width to wrap text automatically
        text width=3.0cm,
        align=center
    },
    arrow/.style={
        thick,
        ->,
        >=Stealth,
        shorten <=1pt,
        shorten >=1pt
    },
}

\definecolor{codegreen}{rgb}{0,0.6,0}
\definecolor{codegray}{rgb}{0.5,0.5,0.5}
\definecolor{codepurple}{rgb}{0.58,0,0.82}
\definecolor{backcolour}{rgb}{0.95,0.95,0.92}

\usepackage{listings}

% Define C style
\lstdefinestyle{cstyle}{
    backgroundcolor=\color{backcolour},
    commentstyle=\color{codegreen},
    keywordstyle=\color{blue}, % Keywords in blue
    numberstyle=\tiny\color{codegray},
    stringstyle=\color{codepurple},
    basicstyle=\ttfamily\footnotesize,
    breakatwhitespace=false,
    breaklines=true,
    captionpos=b,
    keepspaces=true,
    numbers=left,
    numbersep=5pt,
    showspaces=false,
    showstringspaces=false,
    showtabs=false,
    tabsize=4,
    language=C,
    morekeywords={uint32_t, uint64_t, int32_t, int64_t, bool, inline}
}

\lstset{style=cstyle}

% Page layout settings
\geometry{a4paper, margin=1in}

%tabularx environment
\usepackage{tabularx}

%for lists within experience section
\usepackage{enumitem}

% centered version of 'X' col. type
\newcolumntype{C}{>{\centering\arraybackslash}X} 

%to prevent spillover of tabular into next pages
\usepackage{supertabular}
\usepackage{tabularx}
\newlength{\fullcollw}
\setlength{\fullcollw}{0.47\textwidth}

%custom \section
\usepackage{titlesec}				
\usepackage{multicol}
\usepackage{multirow}

%CV Sections inspired by: 
%http://stefano.italians.nl/archives/26
\titleformat{\section}{\large\scshape\raggedright}{}{0em}{}[\titlerule]
\titlespacing{\section}{0pt}{10pt}{10pt}

%for publications
\usepackage[style=authoryear,sorting=ynt, maxbibnames=2]{biblatex}

%Setup hyperref package, and colours for links
\usepackage[unicode, draft=false]{hyperref}
\definecolor{linkcolour}{rgb}{0,0.2,0.6}
\hypersetup{colorlinks,breaklinks,urlcolor=linkcolour,linkcolor=linkcolour}
\addbibresource{citations.bib}
\setlength\bibitemsep{1em}

%for social icons
\usepackage{fontawesome5}

\begin{document}
%----------------------------------------------------------------------------------------
%	TITLE+
%----------------------------------------------------------------------------------------
% Title Page
\begin{titlepage}
    \centering
    \vspace*{2cm}
    \Huge{\textbf{EEE3094 Technical Report}}\\[0.5cm]
    \Large{\textbf{Semester 1 Report}}\\ 
    \Large{Sahas Talasila \textit{230057896}}
    \vfill
\end{titlepage}

% Table of Contents
\tableofcontents
\newpage

\begin{abstract}
    This is a placeholder for the abstract of the report.
\end{abstract}

\section{Introduction}
The Introduction should answer three basic questions:

What is the project about?
This section should be a precise description of the subject matter of the
report. It should also include the way it relates to other work in the
School, if this is appropriate.

Why is it being undertaken?
The reasons for undertaking the project are discussed in this section,
outlining the objectives and importance of the work. Since it is
important not to create a false impression of the outcome of the
project a brief statement is made regarding whether or not the aims of
the project were achieved, and, if not, why not.

How is the subject matter of the project described? – This should be a
`Road-Map' of the dissertation.
This is a `Road-Map' of the dissertation and comprises a short summary
of the contents of the subsequent sections.
It informs the reader where, for example, the methods used, results
obtained, etc. are to be found.

\section{Literature Review}

The Literature Review defines the current state of research in your area and
places your work in context with other work in your project area. It also acts
as the foundation for the comparison of your results with other relevant work
in the `Discussion' chapter in your Individual Project.

The review is a summary of relevant articles (that is, material relevant to the
background to your project, overview of the subject, discussion of any
essential theories, etc.) published in technical journals, conference
proceedings, books, websites, etc. - avoid using very general references. The
majority of references should be books or journal papers; web sites should
appear only occasionally in the list.

Most journals have `Special Issues' on given topic areas. These are good
sources of review material and contain large numbers of references.

However, do not cite a reference unless you have read it. Furthermore, a
Literature Review which only cites Stage I and II lecture notes and text
books is not considered to be adequate.

It is imperative to attribute any information cited in the report to the
source from which it was obtained - failure to reference material properly
could result in plagiarism.

The list of references should be provided in the same format as in IEEE
Transactions or IET Proceedings.

\section{Theoretical Background}

This section should provide the theoretical background to the project.
It should include any relevant equations, derivations, and explanations
of concepts that are essential for understanding the project.

\section{Results}

Results should be presented in a clear and succinct way so that they can
be easily interpreted and also supported by appropriate text which
highlights the significant aspects of the results.

If a design project has been undertaken the results will comprise
measurements, observations and simulations which have been carried
out in order to assess how closely the design correlates to its
specification in terms of its functionality, performance, area, power
dissipation, etc.

In this section the results of your investigation or design are analysed and
compared, if appropriate, to related work in the field.
Again, it is also useful to reiterate the aims of the project so that your
discussion can be put in context. What ensues thereafter depends upon
the project. For example, it may include a comparison of:

\begin{itemize}
  \item Measured results with simulation data, or results outlined in related
technical publications.
  \item Results from different proposed solutions, different implementations
of the same solution – advantages / disadvantages of each.
  \item Design performance against its specification.
\end{itemize}

The discussion may also include a Retrospective Section where,
considering the knowledge and experience you gained from the project,
you may give a brief account of how you would have tackled the project
differently.

Suggestions for extending the work of the project can also be outlined in
this section.

\section{Conclusion}

The conclusions should be very succinct and depending on the project
may contain:

\begin{itemize}
  \item Statements on the deductions obtained from the results.
  \item Comments on the extent to which the original aims of the project were
fulfilled.
  \item Novelty of the solutions proposed - merits and limitations.
  \item Statement of potential applications.
\end{itemize}

\section{References}

The list of references follows immediately after the conclusions.
Each reference should be cited somewhere in your dissertation, and
there should be no citations that refer to references that cannot be found
in the list.
Use IEEE standard for referencing. 

\section{Appendices}

should be used for material that is relevant to the project but
disrupts the flow of the main text, e.g. detailed derivations,
additional graphs, tables of results, program listings, etc.

\end{document}